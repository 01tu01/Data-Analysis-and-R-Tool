\section{主成分分析}
    \begin{enumerate}
        \item
        {\kaishu \textcolor{blue}{解:}}\\
        矩阵$\pmb{\Sigma}$的特征值为$\lambda_1=5,\lambda_2=3,\lambda_3=1$,对应的正交单位化特征向量为
        \[\pmb{e}_1=(1,0,0)^{\mathrm{T}},\pmb{e}_2=\left(0,\frac{\sqrt{2}}{2},\frac{\sqrt{2}}{2}\right)^{\mathrm{T}},\pmb{e}_3=\left(0,\frac{\sqrt{2}}{2},-\frac{\sqrt{2}}{2}\right)^{\mathrm{T}},\]
        则
        \[\begin{cases}
            Y_1=\pmb{e}_1^{\mathrm{T}}\pmb{X}=X_1,\\
            \displaystyle Y_2=\pmb{e}_2^{\mathrm{T}}\pmb{X}=\frac{\sqrt{2}}{2}X_2+\frac{\sqrt{2}}{2}X_3,\\
            \displaystyle Y_3=\pmb{e}_3^{\mathrm{T}}\pmb{X}=\frac{\sqrt{2}}{2}X_2-\frac{\sqrt{2}}{2}X_3.
            \end{cases}\]
        各主成分的方差为\[\mathrm{Var}(Y_1)=\lambda_1=5,\mathrm{Var}(Y_2)=\lambda_2=3,\mathrm{Var}(Y_3)=\lambda_3=1,\]
        各主成分的累计贡献率为
        \begin{align*}
            \psi(Y_1) & =\frac{\lambda_1}{\lambda_1+\lambda_2+\lambda_3}=\frac{5}{9}=55.56\%<85\%,\\
            \psi(Y_2) & =\frac{\lambda_1+\lambda_2}{\lambda_1+\lambda_2+\lambda_3}=\frac{8}{9}=88.89\%>85\%,\\
            \psi(Y_3) & =\frac{\lambda_1+\lambda_2+\lambda_3}{\lambda_1+\lambda_2+\lambda_3}=\frac{9}{9}=100.0\%>85\%.
        \end{align*}
        因为各主成分互不相关,所以各主成分间的相关系数均为0,即
        \[\rho_{(Y_i,Y_j)}=0,\quad i \neq j, \quad i,j=1,2,3.\]
        综上,
        \begin{enumerate}[label=(\arabic*)]
            \item $\pmb{X}$累计贡献率$\geq 85\%$的主成分为$Y_1,Y_2$;
            \item $\mathrm{Var}(Y_1)=5,\mathrm{Var}(Y_2)=3,\mathrm{Var}(Y_3)=1$;
            \item $\rho_{(Y_i,Y_j)}=0,\quad i \neq j, \quad i,j=1,2,3$。
        \end{enumerate}
        \item
        {\kaishu\textcolor{blue}{证:}}\\
        令$P=[p_1,p_2,p_3]$,\begin{align*}
            & P^TX\text{是}X\text{的主成分} \iff p_i\text{是}X\text{的协方差矩阵}\Sigma\text{的特征向量}\iff \Sigma p_i=\lambda_i p_i \iff \\
            & (\Sigma+\sigma^2 I)p_i=(\lambda_i+\sigma^2)p_i \iff p_i+\sigma^2\text{是}Y\text{的协方差矩阵}\Sigma+\sigma^2I\text{的特征向量} \iff P^TY\text{是}Y\text{的主成分}.
        \end{align*}
        \item
        {\kaishu\textcolor{blue}{证:}}\\
        协方差矩阵$\Sigma$的特征值为
        \[\begin{cases}
            \lambda_1=\sigma^2+\sigma_{12}-\sigma_{13}-\sigma_{14}\\
            \lambda_2=\sigma^2-\sigma_{12}+\sigma_{13}-\sigma_{14}\\
            \lambda_3=\sigma^2-\sigma_{12}-\sigma_{13}+\sigma_{14}\\
            \lambda_4=\sigma^2+\sigma_{12}+\sigma_{13}+\sigma_{14}
            \end{cases}\]
        且形式$P^TX$中的$P=[p_1,p_2,p_3,p_4]$如下
        \[\begin{array}{cccc}\left(
            \begin{matrix}
            1/2 & 1/2 & 1/2 & 1/2\\
            1/2 & 1/2 & -1/2 & -1/2\\
            1/2 & -1/2 & 1/2 & -1/2\\
            1/2 & -1/2 & -1/2 & 1/2
            \end{matrix}
            \right)\end{array}\]
        很容易验证:$\Sigma p_i=\lambda_i p_i$,则证明完毕。
        \item
        \code
\begin{lstlisting}
x <- read.table("ex5_4-data.txt", header=TRUE)
std1.x <- scale(x)
rownames(std1.x) <- seq(1,42)
std.x <- as.data.frame(std1.x)
# 样本相关阵出发
prin1 <- princomp(std.x, cor=TRUE)
summary(prin1)
loadings(prin1)
# 样本协方差阵出发
prin2 <- princomp(std.x, cor=FALSE)
summary(prin2)
loadings(prin2)
\end{lstlisting}
        \out
\begin{lstlisting}
# 样本相关阵出发
> summary(prin1)
Importance of components:
                          Comp.1    Comp.2    Comp.3    Comp.4    Comp.5    Comp.6
Standard deviation     1.5270812 1.1627564 1.0870278 0.8587576 0.8340510 0.7390645
Proportion of Variance 0.3331395 0.1931432 0.1688042 0.1053521 0.0993773 0.0780309
Cumulative Proportion  0.3331395 0.5262828 0.6950870 0.8004391 0.8998164 0.9778473
                           Comp.7
Standard deviation     0.39378827
Proportion of Variance 0.02215274
Cumulative Proportion  1.00000000
> loadings(prin1)

Loadings:
   Comp.1 Comp.2 Comp.3 Comp.4 Comp.5 Comp.6 Comp.7
X1  0.253  0.203  0.672         0.520  0.327  0.240
X2 -0.221 -0.529  0.190 -0.755  0.245              
X3 -0.546        -0.107  0.253  0.483  0.224 -0.585
X4 -0.377  0.481 -0.351 -0.332         0.416  0.468
X5 -0.488  0.193  0.226  0.178  0.134 -0.717  0.331
X6 -0.330 -0.587         0.433 -0.227  0.354  0.416
X7 -0.314  0.258  0.564 -0.163 -0.605  0.163 -0.313

               Comp.1 Comp.2 Comp.3 Comp.4 Comp.5 Comp.6 Comp.7
SS loadings     1.000  1.000  1.000  1.000  1.000  1.000  1.000
Proportion Var  0.143  0.143  0.143  0.143  0.143  0.143  0.143
Cumulative Var  0.143  0.286  0.429  0.571  0.714  0.857  1.000

# 样本协方差阵出发
> summary(prin2)
Importance of components:
                          Comp.1    Comp.2    Comp.3    Comp.4    Comp.5    Comp.6
Standard deviation     1.5087921 1.1488307 1.0740090 0.8484728 0.8240620 0.7302131
Proportion of Variance 0.3331395 0.1931432 0.1688042 0.1053521 0.0993773 0.0780309
Cumulative Proportion  0.3331395 0.5262828 0.6950870 0.8004391 0.8998164 0.9778473
                           Comp.7
Standard deviation     0.38907207
Proportion of Variance 0.02215274
Cumulative Proportion  1.00000000
> loadings(prin2)

Loadings:
   Comp.1 Comp.2 Comp.3 Comp.4 Comp.5 Comp.6 Comp.7
X1  0.253  0.203  0.672         0.520  0.327  0.240
X2 -0.221 -0.529  0.190 -0.755  0.245              
X3 -0.546        -0.107  0.253  0.483  0.224 -0.585
X4 -0.377  0.481 -0.351 -0.332         0.416  0.468
X5 -0.488  0.193  0.226  0.178  0.134 -0.717  0.331
X6 -0.330 -0.587         0.433 -0.227  0.354  0.416
X7 -0.314  0.258  0.564 -0.163 -0.605  0.163 -0.313

               Comp.1 Comp.2 Comp.3 Comp.4 Comp.5 Comp.6 Comp.7
SS loadings     1.000  1.000  1.000  1.000  1.000  1.000  1.000
Proportion Var  0.143  0.143  0.143  0.143  0.143  0.143  0.143
Cumulative Var  0.143  0.286  0.429  0.571  0.714  0.857  1.000
\end{lstlisting}
        \summary\\
        根据输出,可以得到
        \begin{enumerate}[label=(\arabic*)]
            \item 样本相关阵与样本协方差阵所得结果的特征根有细微差别,但对于方差的贡献率、累计贡献率、主成分的系数等基本完全一致,差别不大。
            \item 不能,前三个主成分的累计贡献率低于70\%,偏低。
            \item 选择前6个主成分,累计贡献率可达90\%。
        \end{enumerate}
    \end{enumerate}
\clearpage