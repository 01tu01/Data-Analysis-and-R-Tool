\section{因子分析}
    \begin{enumerate}
        \item
        {\kaishu \textcolor{blue}{解:}}
        \begin{enumerate}[label=(\arabic*)]
            \item 先计算$\Sigma$的特征值,为$\displaystyle\lambda_1=\frac{4}{3},\lambda_2=1,\lambda_3=\frac{2}{3}$,对应的单位化后的特征向量为
            \[\begin{cases}
                \displaystyle e_1=\left(0,\frac{\sqrt{2}}{2},\frac{\sqrt{2}}{2}\right)^T\\
                e_2=(1,0,0)^T\\
                \displaystyle e_3=\left(0,-\frac{\sqrt{2}}{2},\frac{\sqrt{2}}{2}\right)^T
                \end{cases}\]
            则\[\Sigma=\hat{A}\hat{A}^{\mathrm{T}}=\begin{array}{ccc}\left(
                \begin{matrix}
                1 & 0 & 0\\
                0 & 1 & 1/3\\
                0 & 1/3 & 1
                \end{matrix}
                \right)\end{array}=\begin{array}{ccc}\left(
                \begin{matrix}
                0 & 1 & 0\\
                \frac{\sqrt{6}}{3} & 0 & -\frac{\sqrt{3}}{3}\\
                \frac{\sqrt{6}}{3} & 0 & \frac{\sqrt{3}}{3}
                \end{matrix}
                \right)\end{array}\begin{array}{ccc}\left(
                \begin{matrix}
                0 & \frac{\sqrt{6}}{3} & \frac{\sqrt{6}}{3}\\
                1 & 0 & 0\\
                0 & -\frac{\sqrt{3}}{3} & \frac{\sqrt{3}}{3}
                \end{matrix}
                \right)\end{array},\]
            累计贡献率为
            \begin{align*}
                \frac{\lambda_1}{\sum\limits_{i=1}^3 \lambda_i} & =\frac{4/3}{3}=\frac{4}{9}<75\%,\\
                \frac{\lambda_1+\lambda_2}{\sum\limits_{i=1}^3 \lambda_i} & =\frac{4/3+1}{3}=\frac{7}{9}>75\%,\\
                \frac{\lambda_1+\lambda_2+\lambda_3}{\sum\limits_{i=1}^3 \lambda_i} & =1>75\%.
            \end{align*}
            累计贡献率大于$75\%$的公因子为$f_1,f_2$,
            则因子模型为
            \[\left\{\begin{array}{cccc}
                X_1=& &f_2 & + \varepsilon_1\\
                X_2=&\frac{\sqrt{6}}{3}f_1 & & + \varepsilon_2\\
                X_3=&\frac{\sqrt{6}}{3}f_1 & & + \varepsilon_3
                \end{array}\right.\]
            \item $f_1,f_2,f_3$的方差贡献分别为$\displaystyle\lambda_1=\frac{4}{3},\lambda_2=1,\lambda_3=\frac{2}{3}$。
            \item 由于$\mathrm{Cov}(X_i,f_j)=a_{ij}$,所以
            \[\begin{array}{ccc}
                \mathrm{Cov}(X_1,f_1)=0, & \mathrm{Cov}(X_1,f_2)=1, & \mathrm{Cov}(X_1,f_3)=0,\\
                \mathrm{Cov}(X_2,f_1)=\frac{\sqrt{6}}{3}, & \mathrm{Cov}(X_2,f_2)=0, & \mathrm{Cov}(X_2,f_3)=-\frac{\sqrt{3}}{3},\\
                \mathrm{Cov}(X_3,f_1)=\frac{\sqrt{6}}{3}, & \mathrm{Cov}(X_3,f_2)=0, & \mathrm{Cov}(X_3,f_3)=\frac{\sqrt{3}}{3}.
                \end{array}\]
        \end{enumerate}
        \item
        \code
\begin{lstlisting}
x <- read.table("ex6_2-data.txt", header=TRUE)
fact <- factanal(x, 3, scores="Bartlett", rotation="varimax"); fact
fact$scores
colMeans(x)
\end{lstlisting}
        \out
\begin{lstlisting}
> fact
Call:
factanal(x = x, factors = 3, scores = "Bartlett", rotation = "varimax")

Uniquenesses:
   X1    X2    X3    X4    X5    X6    X7 
0.030 0.005 0.188 0.005 0.005 0.005 0.241 

Loadings:
   Factor1 Factor2 Factor3
X1  0.116   0.950   0.232 
X2          0.379   0.921 
X3 -0.788   0.280   0.335 
X4          0.817   0.573 
X5  0.982   0.171         
X6  0.975   0.141   0.163 
X7  0.530  -0.482  -0.496 

               Factor1 Factor2 Factor3
SS loadings      2.834   2.073   1.617
Proportion Var   0.405   0.296   0.231
Cumulative Var   0.405   0.701   0.932

Test of the hypothesis that 3 factors are sufficient.
The chi square statistic is 57.11 on 3 degrees of freedom.
The p-value is 2.44e-12

> fact$scores
         Factor1     Factor2     Factor3
 [1,]  1.6529734 -1.65211154  0.09649973
 [2,]  0.5937706  0.49870759 -0.15396797
 [3,] -1.0507634 -0.33507998  0.82724465
 [4,] -1.3602507  0.59342604  1.01466103
 [5,] -0.4106409  0.52779334  0.81728819
 [6,]  0.1351855 -0.85757911 -0.02025507
 [7,]  0.9057612 -1.43130352 -1.35275397
 [8,] -0.7273935  0.47787621  0.82636880
 [9,]  2.1907486  0.54815313  1.76154678
[10,]  1.0709157  1.91452350 -2.38819512
[11,]  0.6707224  0.86157369 -1.19951381
[12,] -0.8950983 -0.02771697  0.38397967
[13,] -0.6271528 -1.02904389 -0.33345128
[14,] -0.1280872 -0.39128734 -1.37282935
[15,] -0.7317347 -2.27447887 -0.44110066
[16,]  0.0582647 -0.64983615  0.71251106
[17,]  0.8989055  0.64761904  1.67682588
[18,] -0.5093322  0.48905989 -0.11862349
[19,]  0.5303530 -1.17243219  0.27036622
[20,] -0.2864957  0.31907958 -1.22262548
[21,]  1.7792655  0.63655739  0.76786294
[22,] -0.9984357  1.18081053 -0.16626248
[23,] -1.0715159 -0.67719910  0.25958371
[24,] -0.5497015  0.71240982 -1.01847110
[25,] -1.1402634  1.09047892  0.37331109

> colMeans(x)
    X1     X2     X3     X4     X5     X6     X7 
7.1000 4.7732 2.3488 9.1524 5.4584 7.1672 2.3460
\end{lstlisting}
        \summary\\
        可以得出因子模型为\[\begin{cases}
            x_1 - 7.100 = 0.116f_1 + 0.950f_2 + 0.232f_3 + \varepsilon_1\\
            x_2 - 4.7732 = 0f_1 + 0.379f_2 + 0.921f_3 + \varepsilon_2\\
            x_3 - 2.3488 = -0.788f_1 + 0.280f_2 + 0.335f_3 + \varepsilon_3\\
            x_4 - 9.1524 = 0f_1 + 0.817f_2 + 0.573f_3 + \varepsilon_4\\
            x_5 - 5.4584 = 0.982f_1 + 0.171f_2 + 0f_3 + \varepsilon_5\\
            x_6 - 7.1672 = 0.975f_1 + 0.141f_2 + 0.163f_3 + \varepsilon_6\\
            x_7 - 2.3460 = 0.530f_1 - 0.482f_2 - 0.496f_3 + \varepsilon_7
            \end{cases}\]
        $f_1$在$x_5,x_6$上的载荷很大,可以将$x_5,x_6$合为一个指标;$f_2$在$x_1,x_4$上载荷较大,可以将$x_1,x_4$合为一个指标;$f_3$在$x_2$上载荷很大;这样就可以将7个指标降成3个因子,方便后续操作处理。
        \item
        \code
\begin{lstlisting}
x <- read.table("ex6_3-data.txt", header=TRUE, row.names = 1)
std.x <- as.data.frame(scale(x))
prin1 <- princomp(std.x, cor = T)
screeplot(prin1, type="lines")
# 通过碎石图判断需要2个因子
fact <- factanal(x, 2, scores="Bartlett", rotation="varimax")
fact
fact$scores
colMeans(x)
\end{lstlisting}
        \out
\begin{lstlisting}
> fact

Call:
factanal(x = x, factors = 2, scores = "Bartlett", rotation = "varimax")

Uniquenesses:
   X1    X2    X3    X4    X5    X6 
0.005 0.014 0.068 0.005 0.032 0.008 

Loadings:
   Factor1 Factor2
X1 0.789   0.610  
X2 0.853   0.508  
X3 0.455   0.852  
X4 0.638   0.768  
X5 0.839   0.514  
X6 0.553   0.828  

               Factor1 Factor2
SS loadings      2.975   2.895
Proportion Var   0.496   0.482
Cumulative Var   0.496   0.978

Test of the hypothesis that 2 factors are sufficient.
The chi square statistic is 9.64 on 4 degrees of freedom.
The p-value is 0.047 
> fact$scores
              Factor1     Factor2
shanghai   1.61488810  3.28060030
nanjing    0.04603819  0.25383057
suzhou     2.71139789 -2.01206700
wuxi       0.78530751 -0.64125138
changzhou -0.33545512 -0.21289765
zhenjiang -0.68960491 -0.13452035
nantong   -0.34352013 -0.11683543
yangzhou  -0.80581667  0.04008038
tai4zhou  -0.76307362 -0.08571657
hangzhou   0.62763381 -0.11163156
ningbo     0.24935155  0.02978948
jiaxing   -0.29215815 -0.24665626
huzhou    -0.87148041  0.03201790
shaoxing  -0.26852669 -0.12364901
zhoushan  -1.13400413  0.02154910
tai1zhou  -0.53097720  0.02735747
> colMeans(x)
         X1          X2          X3          X4          X5          X6 
 1487.42125   685.85438 10874.18750   449.13000    58.87375    62.83937 
\end{lstlisting}
        \summary\\
        可以得出因子模型为
        \[\begin{cases}
            x_1 - 1487.42125 = 0.789f_1 + 0.610f_2 + \varepsilon_1\\
            x_2 - 685.85438 = 0.853f_1 + 0.508f_2 + \varepsilon_2\\
            x_3 - 10874.18750 = 0.455f_1 + 0.852f_2 + \varepsilon_3\\
            x_4 - 449.13000 = 0.638f_1 + 0.768f_2 + \varepsilon_4\\
            x_5 - 58.87375 = 0.839f_1 + 0.514f_2 + \varepsilon_5\\
            x_6 - 62.83937 = 0.553f_1 + 0.828f_2 + \varepsilon_6
            \end{cases}\]
        $f_1$在$x_2,x_5$上的载荷较大,可以将固定资产投资与外贸出口额合成一个指标;$f_2$在$x_3,x_6$上的载荷较大,可以将货运总量与互联网上网人数合成一个指标;这样就可以将6个指标降成2个因子,方便后续操作处理。
        \item
        \code
\begin{lstlisting}
ex6_4 <- function(x, m) {
    p <- nrow(x)
    x.diag <- diag(x)
    sum.rank <- sum(x.diag)
    # 设置行名、列名
    rowname <- paste("X", 1:p, sep = "")
    colname <- paste("Factor", 1:m, sep = "")
    # 构造因子载荷矩阵A,初值设为0
    A <- matrix(0, nrow = p, ncol = m, dimnames = list(rowname, colname))
    # eig包含两个元素,values为特征根,vectors为特征向量
    eig <- eigen(x)
    for (i in 1:m) {
          # 填充矩阵A的值
          A[, i] <- sqrt(eig$values[i]) * eig$vectors[, i]
      }
    # 公共因子的方差
    var.A <- diag(A %*% t(A))
    rowname1 <- c("SS loadings", "Proportion Var", "Cumulative Var")
    # 构造输出结果的矩阵,初值设为0
    result <- matrix(0, nrow = 3, ncol = m, dimnames = list(rowname1, colname))
    for (i in 1:m) {
        # 计算各因子的方差
        result[1, i] <- sum(A[, i]^2)
        # 计算方差贡献率
        result[2, i] <- result[1, i] / sum.rank
        # 累计方差贡献率
        result[3, i] <- sum(result[1, 1:i]) / sum.rank
    }
    method <- c("Principal Component Method")
    # 输出计算结果
    list(method = method, loadings = A, var = cbind(common = var.A, specific = x.diag - var.A), result = result)
}
\end{lstlisting}
        \out\\
        略。\\
        \summary\\
        略。
    \end{enumerate}
\clearpage