\section{相关分析}
\begin{enumerate}
    \item
    \code
\begin{lstlisting}
data <- read.table("ex9_1-data.txt", header=T, row.names = 1);
cor(data)
cor.test(~X1+X2, data=data)
cor.test(~X1+X3, data=data)
cor.test(~X1+X4, data=data)
cor.test(~X1+X5, data=data)
cor.test(~X2+X3, data=data)
cor.test(~X2+X4, data=data)
cor.test(~X2+X5, data=data)
cor.test(~X3+X4, data=data)
cor.test(~X3+X5, data=data)
cor.test(~X4+X5, data=data)
\end{lstlisting}
    \out
\begin{lstlisting}
> cor(data)
          X1          X2        X3        X4          X5
X1 1.0000000  0.33837901 0.5169851 0.8991527  0.72340083
X2 0.3383790  1.00000000 0.2772404 0.2701376 -0.05311005
X3 0.5169851  0.27724044 1.0000000 0.4255092  0.39922122
X4 0.8991527  0.27013758 0.4255092 1.0000000  0.76877136
X5 0.7234008 -0.05311005 0.3992212 0.7687714  1.00000000
> cor.test(~X1+X2, data=data)

	Pearson's product-moment correlation

data:  X1 and X2
t = 1.9365, df = 29, p-value = 0.06261
alternative hypothesis: true correlation is not equal to 0
95 percent confidence interval:
 -0.01813559  0.61855376
sample estimates:
     cor 
0.338379 

> cor.test(~X1+X3, data=data)

	Pearson's product-moment correlation

data:  X1 and X3
t = 3.2524, df = 29, p-value = 0.002901
alternative hypothesis: true correlation is not equal to 0
95 percent confidence interval:
 0.1991218 0.7364213
sample estimates:
      cor 
0.5169851 

> cor.test(~X1+X4, data=data)

	Pearson's product-moment correlation

data:  X1 and X4
t = 11.064, df = 29, p-value = 6.341e-12
alternative hypothesis: true correlation is not equal to 0
95 percent confidence interval:
 0.7995557 0.9506198
sample estimates:
      cor 
0.8991527 

> cor.test(~X1+X5, data=data)

	Pearson's product-moment correlation

data:  X1 and X5
t = 5.6423, df = 29, p-value = 4.266e-06
alternative hypothesis: true correlation is not equal to 0
95 percent confidence interval:
 0.4962694 0.8578487
sample estimates:
      cor 
0.7234008 

> cor.test(~X2+X3, data=data)

	Pearson's product-moment correlation

data:  X2 and X3
t = 1.5539, df = 29, p-value = 0.1311
alternative hypothesis: true correlation is not equal to 0
95 percent confidence interval:
 -0.08549887  0.57508563
sample estimates:
      cor 
0.2772404 

> cor.test(~X2+X4, data=data)

	Pearson's product-moment correlation

data:  X2 and X4
t = 1.5109, df = 29, p-value = 0.1416
alternative hypothesis: true correlation is not equal to 0
95 percent confidence interval:
 -0.09311562  0.56992422
sample estimates:
      cor 
0.2701376 

> cor.test(~X2+X5, data=data)

	Pearson's product-moment correlation

data:  X2 and X5
t = -0.28641, df = 29, p-value = 0.7766
alternative hypothesis: true correlation is not equal to 0
95 percent confidence interval:
 -0.3999240  0.3070076
sample estimates:
        cor 
-0.05311005 

> cor.test(~X3+X4, data=data)

	Pearson's product-moment correlation

data:  X3 and X4
t = 2.5321, df = 29, p-value = 0.01701
alternative hypothesis: true correlation is not equal to 0
95 percent confidence interval:
 0.08380479 0.67767319
sample estimates:
      cor 
0.4255092 

> cor.test(~X3+X5, data=data)

	Pearson's product-moment correlation

data:  X3 and X5
t = 2.3448, df = 29, p-value = 0.02609
alternative hypothesis: true correlation is not equal to 0
95 percent confidence interval:
 0.05227608 0.66017318
sample estimates:
      cor 
0.3992212 

> cor.test(~X4+X5, data=data)

	Pearson's product-moment correlation

data:  X4 and X5
t = 6.4735, df = 29, p-value = 4.383e-07
alternative hypothesis: true correlation is not equal to 0
95 percent confidence interval:
 0.5695918 0.8826672
sample estimates:
      cor 
0.7687714 
\end{lstlisting}
    \summary
    \begin{enumerate}[label=(\arabic*)]
        \item 各指标间的相关系数为
        \begin{table}[H]
            \centering
            \begin{tabular}{|c|c|c|c|c|c|}
                \hline
                & $X_1$ & $X_2$ & $X_3$ & $X_4$ & $X_5$ \\ \hline
                $X_1$ & 1 & 0.33837901 & 0.5169851 & 0.8991527 & 0.72340083 \\ \hline
                $X_2$ & 0.3383790 & 1 & 0.2772404 & 0.2701376 & $-0.05311005$ \\ \hline
                $X_3$ & 0.5169851 & 0.27724044 & 1 & 0.4255092 & 0.39922122 \\ \hline
                $X_4$ & 0.8991527 & 0.27013758 & 0.4255092 & 1 & 0.76877136 \\ \hline
                $X_5$ & 0.7234008 & $-0.05311005$ & 0.3992212 & 0.7687714 & 1 \\ \hline
            \end{tabular}
        \end{table}
        \item $X_1,X_2$显著性检验的$p$值为$0.06261 > 0.05$,则接受原假设,认为$X_1,X_2$不具有显著的线性相关性;
        \item $X_1,X_3$显著性检验的$p$值为$0.002901 < 0.05$,则拒绝原假设,认为$X_1,X_3$具有显著的线性相关性;
        \item $X_1,X_4$显著性检验的$p$值为$6.341\times 10^{-12} < 0.05$,则拒绝原假设,认为$X_1,X_4$具有显著的线性相关性;
        \item $X_1,X_5$显著性检验的$p$值为$4.266\times 10^{-6} < 0.05$,则拒绝原假设,认为$X_1,X_5$具有显著的线性相关性;
        \item $X_2,X_3$显著性检验的$p$值为$0.1311 > 0.05$,则接受原假设,认为$X_2,X_3$不具有显著的线性相关性;
        \item $X_2,X_4$显著性检验的$p$值为$0.1416 > 0.05$,则接受原假设,认为$X_2,X_4$不具有显著的线性相关性;
        \item $X_2,X_5$显著性检验的$p$值为$0.7766 > 0.05$,则接受原假设,认为$X_2,X_5$不具有显著的线性相关性;
        \item $X_3,X_4$显著性检验的$p$值为$0.01701 < 0.05$,则拒绝原假设,认为$X_3,X_4$具有显著的线性相关性;
        \item $X_3,X_5$显著性检验的$p$值为$0.02609 < 0.05$,则拒绝原假设,认为$X_3,X_5$具有显著的线性相关性;
        \item $X_4,X_5$显著性检验的$p$值为$4.383\times 10^{-7} < 0.05$,则拒绝原假设,认为$X_4,X_5$具有显著的线性相关性。
    \end{enumerate}
    \item
    \code
\begin{lstlisting}
# 消除X1影响,X2和X3的偏相关系数
x <- read.table("ex9_1-data.txt",head=TRUE);
x <- x[,2:6]
cor(x)
t.df <- nrow(x) - ncol(x)
ndata <- nrow(x)
nvar <- ncol(x)
r23 = (cor(x)[2, 3] - cor(x)[2, 1] * cor(x)[3, 1]) / (sqrt(1 - cor(x)[2, 1] ^ 2) * sqrt(1 - cor(x)[3, 1] ^ 2))
t <- r23 * sqrt(ndata - nvar) / sqrt(1 - r23 ^ 2)
p1 <- pt(t, t.df)
p2 <- p1 - 0.5
if (any(p2 <= 0)) p <- 2 * p1 else p <- 2 * (1 - p1)
data.frame(parial.coefs23 = r23, t = t, df = t.df, p_value = p)
# 消除X1影响,X4和X5的偏相关系数
r45 = (cor(x)[4, 5] - cor(x)[4, 1] * cor(x)[5, 1]) / (sqrt(1 - cor(x)[4, 1] ^ 2) * sqrt(1 - cor(x)[5, 1] ^ 2))
t <- r45 * sqrt(ndata - nvar) / sqrt(1 - r45 ^ 2)
p1 <- pt(t, t.df)
p2 <- p1 - 0.5
if (any(p2 <= 0)) p <- 2 * p1 else p <- 2 * (1 - p1)
data.frame(parial.coefs45 = r45, t = t, df = t.df, p_value = p)
\end{lstlisting}
    \out
\begin{lstlisting}
> data.frame(parial.coefs23 = r23, t = t, df = t.df, p_value = p)
  parial.coefs23         t df   p_value
1      0.1270064 0.6528951 26 0.5195558

> data.frame(parial.coefs45 = r45, t = t, df = t.df, p_value = p)
  parial.coefs45        t df    p_value
1      0.3915981 2.170076 26 0.03932272
\end{lstlisting}
    \summary\\
    消除$X_1$影响时,$X_2,X_3$的偏相关系数为0.1270064;\\
    消除$X_1$影响时,$X_4,X_5$的偏相关系数为0.3915981。
    \item
    \code
\begin{lstlisting}
x <- read.table("ex9_3-data.txt",head=TRUE,row.names=1);
x <- scale(x);
ca <- cancor(x[,1:2], x[,3:4]); ca
\end{lstlisting}
    \out
\begin{lstlisting}
> ca
$cor
[1] 0.7885079 0.0537397

$xcoef
        [,1]       [,2]
X1 0.1127152 -0.2789099
X2 0.1064583  0.2813576

$ycoef
        [,1]       [,2]
X3 0.1029701 -0.3610078
X4 0.1098775  0.3589657

$xcenter
           X1            X2 
 1.243450e-16 -6.049328e-16 

$ycenter
           X3            X4 
-3.380629e-16 -1.359746e-15 
\end{lstlisting}
    \summary\\
    两对典型变量为
    \[\begin{cases}
        \hat{U}_1 = 0.1127152X_1 + 0.1064583X_2,\\
        \hat{V}_1 = 0.1029701X_3 + 0.1098775X_4.
    \end{cases},\begin{cases}
        \hat{U}_2 = -0.2789099X_1 + 0.3589657X_2,\\
        \hat{V}_2 = -0.3610078X_3 + 0.3589657X_4.
    \end{cases}\]
    \item
    \code
\begin{lstlisting}
# 偏相关系数
ex9_4_partial_cor <- function(without, with, data) {
  cols <- sprintf("X%d", c(without, with));
  x <- cbind(data[,without], data[,with]);
  colnames(x) <- cols;
  rho <- cor(x);
  Mab <- det(rho[-1 * with[1], -1 * with[2]]);
  Maa <- det(rho[-1 * with[1], -1 * with[1]]);
  Mbb <- det(rho[-1 * with[2], -1 * with[2]]);
  r <- Mab / (sqrt(Maa) * sqrt(Mbb));
  return(r);
}
# 复相关系数
ex9_4_multiple_cor <- function(without, with, data) {
  cols <- sprintf("X%d", c(without, with));
  x <- cbind(data[, without], data[, with]);
  colnames(x) <- cols;
  rho <- cor(x);
  Maa <- det(rho[-1 * with[1], -1 * with[1]]);
  r <- sqrt(1 - det(rho) / Maa);
  return(r);
}
\end{lstlisting}
利用题2的前半问数据进行输出
\begin{lstlisting}
data <- read.table("ex9_1-data.txt", header=T, row.names = 1);
source("ex9_4_partial_cor.R");
ex9_4_partial_cor(c(1), c(2,3), data);
source("ex9_4_multiple_cor.R");
ex9_4_multiple_cor(c(1), c(2,3), data);
\end{lstlisting}
    \out
\begin{lstlisting}
> ex9_4_partial_cor(c(1), c(2,3), data);
[1] 0.1270064
> ex9_4_multiple_cor(c(1), c(2,3), data);
[1] 0.3588649
\end{lstlisting}
    \summary\\
    利用题2的前半问数据进行输出,故消除$X_1$影响时,$X_2,X_3$的偏相关系数为0.1270064,$X_2,X_3$的复相关系数为0.3588649。
    \item
    \code
\begin{lstlisting}
ex9_5 <- function(r, n, p, q, alpha = 0.1) {
  m <- length(r);
  Q <- rep(0, m);
  lambda <- 1
  for (k in m:1) {
    lambda <- lambda * (1 - r[k] ^ 2);
    Q[k] <- -log(lambda)
  }
  s <- 0;
  i <- m
  for (k in 1:m) {
    Q[k] <- (n - k + 1 - 1 / 2 * (p + q + 3) + s) * Q[k]
    chi <- 1 - pchisq(Q[k], (p - k + 1) * (q - k + 1))
    if (chi > alpha) {
      i <- k - 1;
      break
    }
    s <- s + 1 / r[k] ^ 2
  }
  i
}
\end{lstlisting}
    \out\\
    略。\\
    \summary\\
    略。
\end{enumerate}