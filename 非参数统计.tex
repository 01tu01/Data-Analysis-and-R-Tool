\section{非参数统计}
    \begin{enumerate}
        \item
        \code
\begin{lstlisting}
library(tseries)
w <- c(9.8,10.1,9.7,9.9,10,10,9.8,9.7,9.8,9.9)
binom.test(sum(w>10), sum(w<10)+sum(w>10), al='two.sided')
\end{lstlisting}
        \out
\begin{lstlisting}
> binom.test(sum(w>10), sum(w<10)+sum(w>10), al='two.sided')

	Exact binomial test

data:  sum(w > 10) and sum(w < 10) + sum(w > 10)
number of successes = 1, number of trials = 8, p-value = 0.07031
alternative hypothesis: true probability of success is not equal to 0.5
95 percent confidence interval:
 0.003159724 0.526509671
sample estimates:
probability of success 
                 0.125
\end{lstlisting}
        \summary\\
        $H_0:$中位数$=10$,不需要调整,$H_1:$中位数$\neq 10$,需要调整,$p$值为$0.07031>0.05$,则接受原假设,认为不需要调整。
        \item
        \code
\begin{lstlisting}
library(tseries)
w <- c(432.00,82.46,46.19,43.14,
       21.59,12.61,23.61,22.20,
       11.31,105.53,83.57,21.77,
       11.95,42.17,6.30,39.03)
binom.test(sum(w>32.32), sum(w<32.32)+sum(w>32.32), al='greater')
\end{lstlisting}
        \out
\begin{lstlisting}
> binom.test(sum(w>32.32), sum(w<32.32)+sum(w>32.32), al='greater')

	Exact binomial test

data:  sum(w > 32.32) and sum(w < 32.32) + sum(w > 32.32)
number of successes = 8, number of trials = 16, p-value = 0.5982
alternative hypothesis: true probability of success is greater than 0.5
95 percent confidence interval:
 0.2786027 1.0000000
sample estimates:
probability of success 
                   0.5 
\end{lstlisting}
        \summary\\
        $H_0:$中位数并无增加,$H_1:$中位数有所增加,$p$值为$0.5982>0.05$,则接受原假设,认为中位数并无增加。
        \item
        \code
\begin{lstlisting}
library(tseries)
x <- c(2.300,2.396,2.100,2.442,1.81,2.007,1.766,1.224,1.411,1.438,0.892,1.167)
D <- numeric()
if(length(x)%%2==0){
	c=length(x)/2
	for(i in 1:c)
		D[i]=x[i]-x[i+c]}
if(length(x)%%2==1){
	c=(length(x)+1)/2
	for(i in 1:c-1)
		D[i]=x[i]-x[i+c]}
binom.test(sum(D>0), sum(D<0)+sum(D>0), al='less')
\end{lstlisting}
        \out
\begin{lstlisting}
> binom.test(sum(D>0), sum(D<0)+sum(D>0), al='less')

	Exact binomial test

data:  sum(D > 0) and sum(D < 0) + sum(D > 0)
number of successes = 6, number of trials = 6, p-value = 1
alternative hypothesis: true probability of success is less than 0.5
95 percent confidence interval:
 0 1
sample estimates:
probability of success 
                     1 
\end{lstlisting}
        \summary\\
        $H_0:M \geq 0$,无上升趋势,$H_1:M < 0$,有上升趋势,$p$值为$1>0.05$,则接受原假设,认为销售量没有单调增加趋势。
        \item
        \code
\begin{lstlisting}
library(tseries)
w <- c(3.6,3.9,4.1,3.6,3.8,3.7,3.4,4.0,3.8,4.1,3.9,4.0,3.8,4.2,4.1)
x <- numeric()
for(i in 1:length(w)){
    if(w[i]-median(w)>0)
	x[i]=1
	if(w[i]-median(w)<=0)
	x[i]=0}
x <- factor(x)
runs.test(x)
\end{lstlisting}
        \out
\begin{lstlisting}
> runs.test(x)

	Runs Test

data:  x
Standard Normal = 1.008, p-value = 0.3134
alternative hypothesis: two.sided
\end{lstlisting}
        \summary\\
        $H_0:$质量的变动是随机的,$H_1:$质量的变动不是随机的,$p$值为$0.3134>0.05$,则接受原假设,认为质量的变动是随机的。
        \item
        \code
\begin{lstlisting}
library(tseries)
w <- c(11,19,17,15,15,16,193)
x <- numeric()
for(i in 1:length(w)){
    if(w[i]-median(w)>0)
	x[i]=1
	if(w[i]-median(w)<=0)
	x[i]=0}
x <- factor(x)
runs.test(x)
\end{lstlisting}
        \out
\begin{lstlisting}
> runs.test(x)

	Runs Test

data:  x
Standard Normal = -0.3638, p-value = 0.716
alternative hypothesis: two.sided
\end{lstlisting}
        \summary\\
        $H_0:$一周内各日的死亡数是随机的,$H_1:$一周内各日的死亡数不是随机的,$p$值为$0.716>0.05$,则接受原假设,认为一周内各日的死亡数是随机的。
        \item
        \code
\begin{lstlisting}
x <- c(54.0,55.1,53.8,54.2,52.1,54.2,55.0,55.8,55.1,55.3)
wilcox.test(x, alternative="two.sided", mu=54, exact=FALSE, conf.level=0.95)
\end{lstlisting}
        \out
\begin{lstlisting}
> wilcox.test(x, alternative="two.sided", mu=54, exact=FALSE, conf.level=0.95)

	Wilcoxon signed rank test with continuity correction

data:  x
V = 34, p-value = 0.1906
alternative hypothesis: true location is not equal to 54
\end{lstlisting}
        \summary\\
        $H_0:$零件质量正常,等于54克,$H_1:$零件质量不正常,不等于54克,$p$值为$0.1906>0.05$,则接受原假设,认为零件质量正常。
        \item
        \code
\begin{lstlisting}
dcxfp <- c(0.6,1.2,2.0,2.4,3.1,4.1,5.0,5.9,7.4,13.9)
pzczdz <- c(9.8,10.2,10.6,13.0,14.0,14.8,15.6,15.6,21.6,24.0)
wilcox.test(pzczdz, dcxfp, alternative="l", paired=FALSE, exact=FALSE)
\end{lstlisting}
        \out
\begin{lstlisting}
> wilcox.test(pzczdz, dcxfp, alternative="l", paired=FALSE, exact=FALSE)

	Wilcoxon rank sum test with continuity correction

data:  pzczdz and dcxfp
W = 96, p-value = 0.9998
alternative hypothesis: true location shift is less than 0
\end{lstlisting}
        \summary\\
        $H_0:$皮质醇增多症的血浆总皮醇测定值高于单纯性肥胖,$H_1:$皮质醇增多症的血浆总皮醇测定值不高于单纯性肥胖,$p$值为$0.9998>0.05$,则接受原假设,认为皮质醇增多症的血浆总皮醇测定值高于单纯性肥胖。
        \item
        \code
\begin{lstlisting}
zf <- c(8.2,10.7,7.5,14.6,6.3,9.2,11.9,5.6,12.8,5.2,4.9,13.5)
jk <- c(4.7,6.3,5.2,6.8,5.6,4.2,6.0,7.4,8.1,6.5)
ansari.test(zf, jk, alternative="two.sided", exact=F, conf.level=0.95)
\end{lstlisting}
        \out
\begin{lstlisting}
> ansari.test(zf, jk, alternative="two.sided", exact=F, conf.level=0.95)

	Ansari-Bradley test

data:  zf and jk
AB = 60.5, p-value = 0.1269
alternative hypothesis: true ratio of scales is not equal to 1
\end{lstlisting}
        \summary\\
        $H_0:$尿酸浓度的变异相同,$H_1:$尿酸浓度的变异不同,$p$值为$0.1269>0.05$,则接受原假设,认为尿酸浓度的变异相同。
        \item
        \code
\begin{lstlisting}
x <- c(55,74,68,80,77,69,57,72,63,52,85,66,71,48,83,78,51,67)
y <- c(41,64,61,70,75,60,53,59,61,48,79,65,70,47,81,69,50,62)
d <- x-y
# 符号检验
binom.test(sum(d>5), sum(d>5)+sum(d<5), al="two.sided")
# Wilcoxon符号秩检验
wilcox.test(x-5, y, alternative="two.sided", paired=T, exact=F, correct=F)
\end{lstlisting}
        \out
\begin{lstlisting}
> # 符号检验
> binom.test(sum(d>5), sum(d>5)+sum(d<5), al="two.sided")

	Exact binomial test

data:  sum(d > 5) and sum(d > 5) + sum(d < 5)
number of successes = 8, number of trials = 17, p-value = 1
alternative hypothesis: true probability of success is not equal to 0.5
95 percent confidence interval:
 0.2298327 0.7218817
sample estimates:
probability of success 
             0.4705882 

> # Wilcoxon符号秩检验
> wilcox.test(x-5, y, alternative="two.sided", paired=T, exact=F, correct=F)

	Wilcoxon signed rank test

data:  x - 5 and y
V = 89, p-value = 0.5516
alternative hypothesis: true location shift is not equal to 0
\end{lstlisting}
        \summary\\
        $H_0:$城市噪音已下降5分贝,$H_1:$城市噪音没有下降5分贝,
        \begin{enumerate}[label=(\arabic*)]
          \item 符号检验的$p$值为$1>0.05$,则接受原假设,认为城市噪音已下降5分贝。
          \item Wilcoxon符号秩检验的$p$值为$0.5516>0.05$,则接受原假设,认为城市噪音已下降5分贝。
        \end{enumerate}
        综上,接受原假设,认为城市噪音已下降5分贝。
        \item
        \code
\begin{lstlisting}
pjz <- c(83.46,49.25,70.21,50.06,43.83,44.03,
         43.59,43.61,42.57,55.71,52.85,
         44.61,45.28,46.37,42.19,42.43)
jj <- c(5118,3148,2574,2216,2186,1759,
        1829,2008,1514,3939,2865,3156,
        1816,2530,2476,2651)
cor.test(pjz, jj, alternative="two.sided", method="spearman")
\end{lstlisting}
        \out
\begin{lstlisting}
> cor.test(pjz, jj, alternative="two.sided", method="spearman")

	Spearman's rank correlation rho

data:  pjz and jj
S = 288, p-value = 0.02156
alternative hypothesis: true rho is not equal to 0
sample estimates:
      rho 
0.5764706 
\end{lstlisting}
        \summary\\
        $H_0:$二者不相关,$H_1:$二者相关,$p$值为$0.02156<0.05$,则拒绝原假设,认为二者相关,相关系数为0.5764706。
        \item
        \code
\begin{lstlisting}
pjz <- c(83.46,49.25,70.21,50.06,43.83,44.03,
         43.59,43.61,42.57,55.71,52.85,
         44.61,45.28,46.37,42.19,42.43)
jj <- c(5118,3148,2574,2216,2186,1759,
        1829,2008,1514,3939,2865,3156,
        1816,2530,2476,2651)
cor.test(pjz, jj, alternative="two.sided", method="kendall", exact=F)
\end{lstlisting}
        \out
\begin{lstlisting}
> cor.test(pjz, jj, alternative="two.sided", method="kendall", exact=F)

	Kendall's rank correlation tau

data:  pjz and jj
z = 2.3412, p-value = 0.01922
alternative hypothesis: true tau is not equal to 0
sample estimates:
      tau 
0.4333333 
\end{lstlisting}
        \summary\\
        $H_0:$二者不相关,$H_1:$二者相关,$p$值为$0.01922<0.05$,则拒绝原假设,认为二者相关,相关系数为0.4333333。
        \item
        \code
\begin{lstlisting}
x <- matrix(c(15,13,20,18), nr=2)
chisq.test(x, correct=F)
\end{lstlisting}
        \out
\begin{lstlisting}
> chisq.test(x, correct=F)

	Pearson's Chi-squared test

data:  x
X-squared = 0.0057171, df = 1, p-value = 0.9397
\end{lstlisting}
        \summary\\
        $H_0:$体育达标水平与性别相互独立,即无关,$H_1:$体育达标水平与性别不独立,即有关,$p$值为$0.9397>0.05$,则接受原假设,认为体育达标水平与性别相互独立,即无关。
    \end{enumerate}
\clearpage